\section{Cheetah vs. [fill in the blank]}
\label{comparisons}

This appendix compares Cheetah with various other template/emdedded scripting
languages and internet development frameworks.  As Cheetah is similar to
Velocity at a superficial level you may also wish to read comparisons between
Velocity and other languages at
\url{http://jakarta.apache.org/velocity/ymtd/ymtd.html}.

%%%%%%%%%%%%%%%%%%%%%%%%%%%%%%%%%%%%%%%%%%%%%%%%%%%%%%%%%%%%%%%%%%%%%%%%%%%%%%%%
\subsection{What features are unique to Cheetah}
\label{comparisons.unique}

\begin{itemize}
\item The {\bf block framework} and everything that goes with it: \code{\#block},
     \code{\#redefine}, \code{\#extend}
\item Cheetah's powerful, yet simple, {\bf caching framework}. See
     \ref{TDL.placeholders.caching} for more information.
\item Cheetah's {\bf Unified Dotted Notation} and {\bf autocalling}. See
     sections \ref{TDL.placeholders.unifiedDottedNotation} and
     \ref{TDL.placeholders.autocalling} for more information.
\item Cheetah's searchList. See section \ref{TDL.placeholders.searchList} for more
     information.
\item Cheetah's \code{\#raw} directive.  See section \ref{directives.raw} for more
     information.
\item Cheetah's \code{\#slurp} directive.  See section \ref{directives.slurp} for more
     information.
\item Cheetah's tight integration with {\bf Webware for Python}.  See section
     \ref{webware} for more information.
\item Cheetah's {\bf SkeletonPage framework}.  See section
     \ref{webware.inheritance.skeletonPage} for more information.
\item Cheetah's ability to mix PSP style code with Cheetah Template Definition
     Language syntax.  See section \ref{customizing.plugins} for more
     information.  Because of Cheetah's design and Python's flexibility it is
     relatively easy to extend Cheetah's syntax with syntax elements from almost
     any other template or embedded scripting language.
\end{itemize}

%%%%%%%%%%%%%%%%%%%%%%%%%%%%%%%%%%%%%%%%%%%%%%%%%%%%%%%%%%%%%%%%%%%%%%%%%%%%%%%%
\subsection{Cheetah vs. Velocity}
\label{comparisons.velocity}

For a basic introduction to Velocity please visit
\url{http://jakarta.apache.org/velocity}.


Velocity is older, has a larger user-base, and has better examples and
docs at the moment. Cheetah, however, has a number of advantages over
Velocity:
\begin{itemize}
\item Cheetah is written in Python. Thus, it's easier to use and extend.
\item Cheetah's syntax is cleaner, more flexible and is closer to Python's.
     Here's an example
\begin{verbatim}
#for $i in range(15)
$i
#end for 
\end{verbatim}
     instead of 
\begin{verbatim}
#foreach($i in range(15))  
$i
#end
\end{verbatim}
     which means the following example is possible Cheetah, but not
     Velocity:
\begin{verbatim}
#for $key, $val in $myDict
$key, $val
#end for 
\end{verbatim}


\item Cheetah has a powerful caching mechanism.  Velocity has no equivalent.
\item It's far easier to add data/objects into the namespace where \$placeholder
     values are extracted from in Cheetah.  Velocity calls this namespace a 'context'.
     Contexts are dictionaries/hashtables. You can put anything you want into a
     context, BUT you have to use the .put() method to populate the context as
     in this example:

\begin{verbatim}
VelocityContext context1 = new VelocityContext();
context1.put("name","Velocity");
context1.put("project", "Jakarta");
context1.put("duplicate", "I am in context1");
\end{verbatim}
     
     Cheetah takes a different approach. Rather than require you to manually
     populate the 'namespace' like Velocity, Cheetah will accept any existing
     Python object or dictionary AS the 'namespace'.  Furthermore, Cheetah
     allows you to specify a list namespaces that will be searched in sequence
     to find a varname-to-value mapping.  This searchList can be extended at
     run-time.
     
     If you add 'foo' object to the searchList and the 'foo' has an attribute
     called 'bar' you can simply type \$bar in the template rather than \$foo.bar.
     If the second item in the searchList is this dictionary {'spam':1234,
       'parrot':666} then \$spam's value will be found in the dict rather than in
     the 'foo' object at the start of the searchList.

\item Cheetah has better whitespace handling around \#directive tags
\item Cheetah has an extension to the \#macro syntax that makes it easier to call
     macros that accept large strings as arguments: e.g. a macro that
     pretty-prints a chunk of source code.
\item Cheetah integrates tightly with Webware.  Velocity doesn't integrate as easily
     with Turbine.
\item Cheetah has a plugin that enables PSP-style coding to be freely mixed in with
     the Cheetah syntax. Velocity doesn't.
\item It is easy to add new \#directives to Cheetah. You can't do this easily in
     Velocity.
\item In Cheetah, the tokens that are used to signal the start of \$placeholders and
     \#directives are configurable. You can set them to any character sequences,
     not just \$ and \#.
\end{itemize}


%%%%%%%%%%%%%%%%%%%%%%%%%%%%%%%%%%%%%%%%%%%%%%%%%%%%%%%%%%%%%%%%%%%%%%%%%%%%%%%%
\subsection{Cheetah vs. WebMacro}
\label{comparisons.webmacro}

For a basic introduction to WebMacro please visit
\url{http://webmacro.org}.

The points discussed in section \ref{comparisons.velocity} also apply to the
comparison between Cheetah and WebMacro.  For further differences please refer
to \url{http://jakarta.apache.org/velocity/differences.html}.

%%%%%%%%%%%%%%%%%%%%%%%%%%%%%%%%%%%%%%%%%%%%%%%%%%%%%%%%%%%%%%%%%%%%%%%%%%%%%%%%
\subsection{Cheetah vs. Zope's DTML}
\label{comparisons.dtml}

For a basic introduction to DTML please visit
\url{http://www.zope.org/Members/michel/ZB/DTML.dtml}.

\begin{itemize}
\item Cheetah is faster than DTML.
\item Cheetah does not use XML style tags, DTML does.  Thus, Cheetah tags are
     visible in rendered HTML output if something goes wrong.
\item DTML can only be used with ZOPE for web development, while Cheetah can be
     used as a standalone tool for any purpose.
\item Cheetah's documentation is more complete than DTML's.
\item Cheetah's learning curve is shorter than DTML's.
\item DTML has no equivalent of Cheetah's blocks, caching framework, macros,
     unified dotted notation, auto-calling, searchList, \code{\#raw} directive,
     \code{\#stop} directive
\end{itemize}

Here are some examples of syntax differences between DTML and Cheetah:
\begin{verbatim}
<ul>
<dtml-in frogQuery>
 <li><dtml-var animal_name></li>
</dtml-in>
</ul>
\end{verbatim}

\begin{verbatim}
<ul>
#for $animal_name in $frogQuery
 <li>$animal_name</li>
#end for
</ul>
\end{verbatim}

\begin{verbatim}
<dtml-if expr="monkeys > monkey_limit">
  <p>There are too many monkeys!</p>
<dtml-elif expr="monkeys < minimum_monkeys">
  <p>There aren't enough monkeys!</p>
<dtml-else>
  <p>There are just enough monkeys.</p>
</dtml-if>
\end{verbatim}

\begin{verbatim}
#if $monkeys > $monkey_limit
  <p>There are too many monkeys!</p>
#else if $monkeys < $minimum_monkeys
  <p>There aren't enough monkeys!</p>
#else
  <p>There are just enough monkeys.</p>
#end if
\end{verbatim}

\begin{verbatim}
<table>
<dtml-in expr="objectValues('File')">
  <dtml-if sequence-even>
    <tr bgcolor="grey">
  <dtml-else>
    <tr>
  </dtml-if>    
  <td>
  <a href="&dtml-absolute_url;"><dtml-var title_or_id></a>
  </td></tr>
</dtml-in>
</table>
\end{verbatim}

\begin{verbatim}
<table>
#set $evenRow = 0
#for $file in $objectValues('File')
  #if $evenRow
    <tr bgcolor="grey">
    #set $evenRow = 0
  #else
    <tr>
    #set $evenRow = 1
  #end if
  <td>
  <a href="$file.absolute_url">$file.title_or_id</a>
  </td></tr>
#end for
</table>
\end{verbatim}

%%%%%%%%%%%%%%%%%%%%%%%%%%%%%%%%%%%%%%%%%%%%%%%%%%%%%%%%%%%%%%%%%%%%%%%%%%%%%%%%
\subsection{Cheetah vs. Zope Page Templates}
\label{comparisons.zpt}

For a basic introduction to Zope Page Templates please visit
\url{http://www.zope.org/Documentation/Articles/ZPT2}.

%%%%%%%%%%%%%%%%%%%%%%%%%%%%%%%%%%%%%%%%%%%%%%%%%%%%%%%%%%%%%%%%%%%%%%%%%%%%%%%%
\subsection{Cheetah vs. PSP, PHP, ASP, JSP, Embperl, etc.}
\label{comparisons.pspEtc}

[This section is under construction.]

\begin{description}
\item[Webware's PSP Component] -- \url{http://webware.sourceforge.net/Webware/PSP/Docs/}
\item[PHP Home Page] -- \url{http://www.php.net/}
\item[Tomcat JSP Information] -- \url{http://jakarta.apache.org/tomcat/index.html}
\item[ASP Information at ASP101] -- \url{http://www.asp101.com/}
\item[Embperl] -- \url{http://perl.apache.org/embperl/}
\end{description}


Here's a basic Cheetah example:
\begin{verbatim}
<TABLE>
#for $client in $service.clients
<TR>
<TD>$client.surname, $client.firstname</TD>
<TD><A HREF="mailto:$client.email" >$client.email</A></TD>
</TR>
#end for
</TABLE>
\end{verbatim}

Compare this with PSP:

\begin{verbatim}
<TABLE>
<% for client in service.clients(): %>
<TR>
<TD><%=client.surname()%>, <%=client.firstname()%></TD>
<TD><A HREF="mailto:<%=client.email()%>"><%=client.email()%></A></TD>
</TR>
<%end%>
</TABLE>
\end{verbatim}


