\section{The Template Definition Language}

%% @@MO:
% This chapter is for any information related to the TDL as a whole.
% Since the vocabulary page already says the TDL consists of $P and #D,
% and the details are fully explained in the next three chapters
% ("Placeholder Tags", "Directive Tags", "Functions and Macros"), I don't
% know what if anything needs to go in this chapter.

%% @@TR: I disagree.

{\bf Template definitions} are text strings, or files, that have been marked up
with Cheetah's {\bf Template Definition Language} for special processing.  This
language is not a general purpose programming language like Python.  Rather,
it's a mini-language that was designed to make HTML-generation, and
code-generation in general, easy enough for non-programmers to understand and
programmers to love.  It is purposefully limited so complex tasks are left to
Python code, where they belong.

Cheetah's Template Definition Language has 2 primary types of tags: {\bf
  placeholders} and {\bf directives}. Placeholder tags begin with a dollar sign
(\code{\$varName}) and are replaced with the value of the variable they refer to
when the template is filled. Directives begin with a hash character (\#) and are
used for everything else: for loops, conditional blocks, comments, includes, and
other advanced features. The following chapters deal with these tags in detail.

