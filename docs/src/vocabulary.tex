\section{Vocabulary}
\label{vocabulary}

To avoid repetition, all Cheetah-specific terms are defined here.
Abbreviations are in parentheses after the term.  If you are
a new user, you do not need to learn all of them now.  Just come back to this
page when you encounter a term you don't recognize.

%%%%%%%%%%%%%%%%%%%%%%%%%%%%%%%%%%%%%%%%%%%%%%%%%%%%%%%%%%%%%%%%%%%%%%%%%%%%%%%%
\subsection{Template object terms}

\begin{description}

\item{Template Object (TO, template)}  An instance of the class
     \code{Cheetah.Template.Template} or one of its subclasses.  A 
     Template Object contains a Template Definition and a SearchList.

\item{Template Definition (TD)}  A string which is the source of the template.

\item{Template Definition Language (TDL)}  The markup language used for
     Template Definitions.  The language defines Placeholder Tags and
     Directive Tags for the \bf{changeable data} in the template.

\item{Search List (SL)}  A list of Python objects which will be searched to find
     values for the placeholders in the template.  Each object is called a
     \bf{Namespace} and must contain attributes, keys or subscripts to
     search in.  

\item{Filled Template (FT)}  A string which is the result of filling in the
     changeable data in the template with their current values.
     Creating a Filled Template is called ``\bf{invoking} the
     template'' or ``\bf{filling} the template''.

\item{Parent Template Definition (Parent TD)}  The Template Definition that is
     being extended when the \code{#extend} directive is used.

\end{description}

%%%%%%%%%%%%%%%%%%%%%%%%%%%%%%%%%%%%%%%%%%%%%%%%%%%%%%%%%%%%%%%%%%%%%%%%%%%%%%%%
\subsection{Tag and value terms}

\begin{description}

\item{Directive Tag (#D, directive)}  All tags that begin with ``#''; e.g.,
     \code{#if}.  These are commands or \bf{macros (#M)}.  (See the
     ``Functions and Macros'' chapter for a definition of macros.)
     \bf{Display logic tags} is an informal term for the tags 
     \code{#if} and \code{#for}.

\item{Placeholder Tag (\$P, placeholder)}  All tags that begin with ``\$''.
     These contain a \bf{Placeholder Name} which is looked up in the Search
     List.  A Placeholder Name may consist of one or more \bf{Identifiers}
     separated by periods, like Python's dotted notation.  Identifiers may
     be followed by \bf{arguments} in ``()'' or ``[]''.  See the section
     ``Placeholder Tags'' for more information.

\item{function/method (FM)}  A Python function or method.  These are the only
     types which may be \bf{autocalled}.  See the section ``Placeholder 
     values'' for an explanation of autocalling.  Note: other callable objects
     (especially classes and instances) are \em{not} FMs and are never
     autocalled.

\end{description}

