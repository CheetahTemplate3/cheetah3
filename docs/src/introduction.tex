\section{Introduction}
\label{intro}


%%%%%%%%%%%%%%%%%%%%%%%%%%%%%%%%%%%%%%%%%%%%%%%%%%%%%%%%%%%%%%%%%%%%%%%%%%%%%%%%
\subsection{Who should read this Guide?}

This Users Guide is for those wishing an overview, tutorial and reference for
the Cheetah template system.  A basic knowledge of Python is assumed.
This Guide also contains examples of integrating Cheetah with Webware for
Python, a web application development framework.  Knowledge of Webware is not
assumed, but of course you will have to learn Webware from its own 
documentation in order to build a Webware + Cheetah site.

Further resources on Python and Webware are in the appendix ``Links''.
(Not written yet.)

%%%%%%%%%%%%%%%%%%%%%%%%%%%%%%%%%%%%%%%%%%%%%%%%%%%%%%%%%%%%%%%%%%%%%%%%%%%%%%%%
\subsection{What is Cheetah?}
\label{intro.whatIs}

Cheetah is a Python-based template engine and code generator.  

A template is like a form letter: it contains text with placeholders for the
changeable data that needs to be filled in.  Python's \code{\%} operator
provides a primitive tool for working with templates:

\begin{verbatim}
>>> print """
... <HTML>
... <HEAD><TITLE>%(title)s</TITLE></HEAD>
... <BODY>
... %(contents)s
... </BODY>
... </HTML>""" % {'title': 'Hello World Example', 'contents': 'Hello World!'}

<HTML>
<HEAD><TITLE>Hello World Example</TITLE></HEAD>
<BODY>
Hello World!
</BODY>
</HTML>
\end{verbatim}

Cheetah is like the \code{\%} operator on steroids.  It's fast, flexible and
powerful.  (The ``philosophy'' section below describes some of Cheetah's
advanced features.) 

Cheetah was designed for Internet development and provides integration with
{\bf Webware for Python} (\url{http://webware.sourceforge.net/}).  Together
with Webware, Cheetah is a compelling alternative to ASP, JSP, PHP and PSP for
building dynamic web sites.

Here is Cheetah's equivalent of the above example:

\begin{verbatim}
>>> from Cheetah.Template import Template
>>> templateDef = """
... <HTML>
... <HEAD><TITLE>$title</TITLE></HEAD>
... <BODY>
... $contents
... </BODY>
... </HTML>"""
>>> nameSpace = {'title': 'Hello World Example', 'contents': 'Hello World!'}
>>> templateObj = Template(templateDef, nameSpace)
>>> print templateObj
 
<HTML>
<HEAD><TITLE>Hello World Example</TITLE></HEAD>
<BODY>
Hello World!
</BODY>
</HTML>
\end{verbatim}


But you can do much more with Cheetah than just this.



%%%%%%%%%%%%%%%%%%%%%%%%%%%%%%%%%%%%%%%%%%%%%%%%%%%%%%%%%%%%%%%%%%%%%%%%%%%%%%%%
\subsection{What is the philosophy behind Cheetah?}
\label{intro.aims}
Cheetah aims:
\begin{itemize}
\item {\bf to make it easy to separate content, graphic design, and program code.}
     
     A clean separation makes it easier for a team of content writers,
     HTML/graphic designers, and programmers to work together without stepping
     on each other's toes and polluting each other's work.  The HTML framework
     and the content it contains are two separate things, and analytical
     calculations (program code) is a third thing.  Each team member should be
     able to concentrate on their specialty and to implement their changes
     without having to go through one of the others (i.e., the dreaded
     ``webmaster bottleneck'').

     Other advantages to separation include:
     \begin{itemize}
     \item faster development time.
     \item HTML and program code that is easier to understand and maintain.
     \item content that can be displayed in a variety of non-HTML formats such
          as PDF.
     \item highly modular, flexible, and reusable site architectures.       
     \end{itemize}
     
\item {\bf to make it easy to integrate content, graphic design, and program code.}
     
     While it should be easy to develop content, graphics and program
     code separately, it should be easy to integrate them together into a 
     website.  In particular, it should be easy:

     \begin{itemize}
     \item for \bf{programmers} to create reusable components and functions
          that are accessible and understandable to designers.
     \item for \bf{designers} to mark out placeholders for content and 
          dynamic components in their templates.
     \item for \bf{designers} to soft-code aspects of their design that are
          either repeated in several places or are subject to change.
     \item for \bf{designers} to extend and customize existing templates and
          thus minimize duplication of effort and code.
     \item and, of course, for \bf{content writers} to use the templates that
          designers have created.
     \end{itemize}

     
\item {\bf to provide template designers with a small set of `Display Logic'
       programming structures such as conditional blocks and for loops}
     
     Graphic designers often do tasks that would be easier, faster, and less
     error prone if they had access to {\bf conditional blocks} and {\bf for
       loops}.  However, a full programming language would be overkill for these
     simple tasks and most designers don't have the time or desire to learn
     one. Complex logic belongs in Python code.
     
\item {\bf to be equally well-suited for HTML, SGML, XML, SQL, Postscript, form
       email, LaTeX, or any other text-based format.}
     
     Although Cheetah was designed with dynamic websites and web applications
     in mind, it is not HTML-specific.
     
\item {\bf to be efficient, flexible and extendable.}
     
\end{itemize}

Cheetah achieves these aims by:

\begin{itemize}     
     
\item blending the power and flexibility of Python with the simplicity of a
     small Template Definition Language (TDL) that non-programmers can
     understand.
     
\item giving template designers a simple way of accessing Python variables,
     objects, and functions in their templates.
     
\item providing a modular, object-orientated framework that makes it easy to
     create and maintain large websites.
     
\item compiling `Template Definitions' into native Python code at startup.
     The code is then executed for each request.  This approach is
     dramatically faster than the string-substitution approach used by many
     templating engines.

\item providing a very simple--yet powerful--caching mechanism that can
     significantly increase the responsiveness of a dynamic website.

\end{itemize}

Cheetah's Template Definition Language is robust but limited.  It's
\bf{robust} because your templates can have named portions (``blocks'') which
you can override individually, conditional portions (displayed only if X is
true), and repeated portions (equivalent to ``for'' loops).  While the
\code{\%} operator looks up placeholders in a single flat dictionary, Cheetah
looks them up in a user-supplied list of objects (the SearchList).  The
searchable objects can be instances or dictionaries--Cheetah will search
attributes and/or keys as appropriate.  A neat trick is to subclass the
template object, assign your data values as attributes, and put `self' in the
SearchList!  But the language is \bf{limited} because it's meant to complement
Python, not replace it.  A placeholder may be a method call or an
attribute.subattribute, but it may not be an expression with operators.
Complex calculations should be done outside the template and the results
assigned to simple attributes in the SearchList.



%%%%%%%%%%%%%%%%%%%%%%%%%%%%%%%%%%%%%%%%%%%%%%%%%%%%%%%%%%%%%%%%%%%%%%%%%%%%%%%%
\subsection{Why is it called Cheetah?}
\label{intro.name}

Cheetah is fast, flexible, agile and graceful - like its namesake. 

[INSERT: picture of a cheetah.]


%%%%%%%%%%%%%%%%%%%%%%%%%%%%%%%%%%%%%%%%%%%%%%%%%%%%%%%%%%%%%%%%%%%%%%%%%%%%%%%%
\subsection{Who developed Cheetah?}
\label{intro.developers}

Cheetah is one of several templating frameworks that grew out of a `templates'
thread on the Webware For Python email list.  Tavis Rudd, Mike Orr, Chuck
Esterbrook, Ian Bicking and Tom Schwaller are the core developers.

%%%%%%%%%%%%%%%%%%%%%%%%%%%%%%%%%%%%%%%%%%%%%%%%%%%%%%%%%%%%%%%%%%%%%%%%%%%%%%%%
\subsection{How mature is Cheetah?}
\label{intro.mature}

Cheetah is alpha software as some aspects of its design are
still subject to change and the Users' Guide is incomplete.
We plan to release a stable version later in 2001.



Here's a summary of known issues and aspects of the design that are in flux,
as well as bugs in the implementation.

\begin{itemize}

\item The \#include directive is not working with relative path file includes
     when used with Webware. This should be resolved soon.

\item The \#include directive is being reworked to monitor for changes in the
     included file at run-time. It currently does the include once-off at
     compile-time.

\item The parsing of macros is currently being worked on to make it more
     robust.  However, the macro interface is stable.

\item The current rules for the autocall feature are brand new and need
     testing.

\item Straightforward use of function arguments, macro arguments, and
     #if and #for expressions is supported, but the limits of what kind of
     expressions are allowed has not been well defined yet, and unanticipated
     expression styles may cause breakage.  Macro arguments will soon follow
     the same rules as function arguments, which will only be a minor change.

\item Quoted string literals in arguments/expressions are supposed to follow
     the same rules as Python, but there are some bugs.  
     'string\'s' stops at the second '.  (Believed to be an easy fix.)
     """ ... """ and ''' ... ''' fail when spanning multiple lines.
     (Appears to be a bug in Python's Tokenizer module.)
     Unknown whether letter prefixes (r"" and u"") and backslash escapes
     are/will be supported.

\item There are other issues which will be added to this list as they are
     articulated in a summarizable form.

\end{itemize}

What \em{is} stable and permanent about Cheetah's API?  This list is not
articulated yet, but probably includes:

\begin{itemize}

\item  Much of the vocabulary.  We are trying to use consistent terminology in
      this Guide.

\item  The constructor and methods of the Template class, the Search List, 
      and the cheetah-compile program.

\item  Placeholder tags:  \$placeholderName or \$placeholderName.  Case
      sensitivity.  Uniform Dotted Notation (in \$a.b, 'b' can be an attribute
      or key of 'a'.)  Trailing periods are not part of the placeholder.
      Autocalling.  The caching rules are "stable unless something comes up".
      Fuction arguments in (); keys/subscripts in [].

\item  Directive tag syntax, /# or newline to end a directive tag, #end tags 
      to terminate multiline directives.  The existence and intent of ##,
      #comment, #raw, #include, #cache, #for, #if, #set, #block, #redefine,
      #slurp.  Macro defining and calling is pretty stable, except for the
      argument syntax.  #longmacro will remain for calling macros with
      multi-line arguments, although the argument syntax may change.

\end{itemize}


%%%%%%%%%%%%%%%%%%%%%%%%%%%%%%%%%%%%%%%%%%%%%%%%%%%%%%%%%%%%%%%%%%%%%%%%%%%%%%%%
\subsection{Where can I get releases?}
\label{intro.releases}

Download Cheetah releases from
\url{http://CheetahTemplate.sourceforge.net}

%%%%%%%%%%%%%%%%%%%%%%%%%%%%%%%%%%%%%%%%%%%%%%%%%%%%%%%%%%%%%%%%%%%%%%%%%%%%%%%%
\subsection{Where can I get news?}
\label{intro.news}

News and updates can be obtained from the the Cheetah website:
\url{http://CheetahTemplate.sourceforge.net}

Cheetah discussions take place on the list
\email{cheetahtemplate-discuss@lists.sourceforge.net}.

If you encounter difficulties, or are unsure about how to do something,
please post a detailed message to the list.

%%%%%%%%%%%%%%%%%%%%%%%%%%%%%%%%%%%%%%%%%%%%%%%%%%%%%%%%%%%%%%%%%%%%%%%%%%%%%%%%
\subsection{How can I contribute?}
\label{intro.contribute}

Cheetah is the work of many volunteers.  If you use Cheetah please share your
experiences, tricks, customizations, and frustrations.

\subsubsection{Bug reports and patches}

If you think there is a bug in Cheetah, send a message to the email list
with the following information:

\begin{enumerate}
\item a description of what you were trying to do and what happened
\item all tracebacks and error output
\item your version of Cheetah
\item your version of Python
\item your operating system
\item whether you have changed anything in the Cheetah installation
\end{enumerate}

\subsubsection{Example sites and tutorials}
If you're developing a website with Cheetah, please send a link to the
email list so we can keep track of Cheetah sites.  Also, if you discover
new and interesting ways to use Cheetah please share your experience and
write a quick tutorial about your technique.

\subsubsection{Macro libraries and function libraries}
We hope to build up a framework of macros libraries (see section
\ref{macros.libraries}) to distribute with Cheetah and would appreciate
any contributions.

\subsubsection{Test cases}
Cheetah is packaged with a regression testing suite that is run with each
new release to ensure that everything is working as expected and that recent
changes haven't broken anything.  The test cases are in the Cheetah.Tests
module.  If you find a reproduceable bug please consider writing a test case
that will pass only when the bug is fixed.  Send any new test cases to the email
list with the subject-line ``new test case for Cheetah.''

\subsubsection{Publicity}
Help spread the word ... recommend it to others, write articles about it, etc.

%%%%%%%%%%%%%%%%%%%%%%%%%%%%%%%%%%%%%%%%%%%%%%%%%%%%%%%%%%%%%%%%%%%%%%%%%%%%%%%%
\subsection{Acknowledgements}
\label{intro.acknowledgments}
    
We'd like to thank the following people for contributing valuable advice, code
and encouragement: Geoff Talvola, Jay Love, Terrel Shumway, Sasa Zivkov, Arkaitz
Bitorika, Jeremiah Bellomy, Baruch Even, Paul Boddie, Stephan Diehl, and Geir
Magnusson.
    
The Velocity, WebMacro and Smarty projects provided inspiration and design
ideas.  Cheetah has benefitted from the creativity and energy of their
developers. Thank you.

%%%%%%%%%%%%%%%%%%%%%%%%%%%%%%%%%%%%%%%%%%%%%%%%%%%%%%%%%%%%%%%%%%%%%%%%%%%%%%%%
\subsection{License}
\label{intro.license}

Cheetah is released for unlimited distribution under the terms of the
Python license.

