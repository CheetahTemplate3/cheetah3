\section{Introduction}
\label{intro}

%%%%%%%%%%%%%%%%%%%%%%%%%%%%%%%%%%%%%%%%%%%%%%%%%%%%%%%%%%%%%%%%%%%%%%%%%%%%%%%%
\subsection{What is Cheetah?}
\label{intro.whatIs}

Cheetah is a Python-based templating framework.  Templates are about
{\em separation}: separating content from logic, and separating parts that are
the programmer's responsibility from those that are the content manager's
responsibility.  A template is like a form letter, containing the parts of a
document that don't change, with placeholders for the parts that do change.
You can do a primitive sort of substitution with Python's "%" operator:

\begin{verbatim}
>>> dic = {'who': 'dog', 'what': 'lazy bum'}
>>> print "Your %(who)s is a %(what)s." % dic
Your dog is a lazy bum.
\end{verbatim}

Cheetah is like a "%" operator on steroids.  Not only is it faster, but you can
do lots more cool things with it.  You can name a portion of a template and
override it later (e.g., to implement a common header/footer with changing body
content, or to nest templates within templates) conditionally display a
portions, display it repeatedly with different substitutions each time, and
more.  The template object is a "server", meaning you can invoke it multiple 
times, changing the substitution values between each invocation.  In fact, 
Cheetah is optimized for this: it compiles the template definition, so each
invocation is faster than ordinary string interpolation.  Cheetah can optionally
cache values that are not expected to change, making it even more faster.
Your substitution pool is not limited to a flat dictionary: Cheetah
will take a list of objects and search each object's attributes and/or keys in
turn for the needed identifier.  (Normally, however, your template will be
wrapped in a subclass of Cheetah.Template, and the object to search will be
'self'.)  
[DELETE? You can even call an {\code attribute.subattribute} or a
\{code method(with, arguments)} ], 
but we'll get into the details later.  For now, let's look at the Cheetah
equivalent of the above simple example:

\begin{verbatim}
>>> from Cheetah.Template import Template
>>> td = "Your $who is a $what."       # The template definition.
>>> dic = {'who': 'dog', 'what': 'lazy bum'}
>>> t = Template(td, dic)
>>> print t
Your dog is a lazy bum.
>>> dic['what'] = 'brilliant poodle'
>>> print t
Your dog is a brilliant poodle.
\end{verbatim}

Even though the template definition language (TDL) is more complex than the "%"
operator's syntax, its feature set has purposely been limited to a few pieces
of display logic.  Analytical logic belongs in Python code that sets simple
attributes for the template to read.  Thus, a placeholder may be a method call
or an attribute.subattribute, but it may not be an expression with operators.
If you need operators, you should be using Python code.

Although Cheetah was designed to help build dynamic web sites, it's not limited
to producing HTML.  It can produce any text-based format: XML, SQL, Postscript,
e-mail message, LaTeX, etc.



%%%%%%%%%%%%%%%%%%%%%%%%%%%%%%%%%%%%%%%%%%%%%%%%%%%%%%%%%%%%%%%%%%%%%%%%%%%%%%%%
\subsection{Why is it called Cheetah?}
\label{intro.name}

Cheetah is fast, flexible, agile and graceful - like its namesake. 


%%%%%%%%%%%%%%%%%%%%%%%%%%%%%%%%%%%%%%%%%%%%%%%%%%%%%%%%%%%%%%%%%%%%%%%%%%%%%%%%
\subsection{Who developed Cheetah?}
\label{intro.developers}

Cheetah is one of several templating frameworks that grew out of a 'templates'
thread on the 'Webware For Python' email list.  Tavis Rudd, Mike Orr, Chuck
Esterbrook, Ian Bicking and Tom Schwaller are the core developers.

%%%%%%%%%%%%%%%%%%%%%%%%%%%%%%%%%%%%%%%%%%%%%%%%%%%%%%%%%%%%%%%%%%%%%%%%%%%%%%%%
\subsection{How mature is Cheetah?}
\label{intro.mature}

Cheetah is alpha level software because the user interface is still evolving.

%%%%%%%%%%%%%%%%%%%%%%%%%%%%%%%%%%%%%%%%%%%%%%%%%%%%%%%%%%%%%%%%%%%%%%%%%%%%%%%%
\subsection{Where can I get releases?}
\label{intro.releases}

Cheetah releases can be downloaded from
\url{http://CheetahTemplate.sourceforge.net}

%%%%%%%%%%%%%%%%%%%%%%%%%%%%%%%%%%%%%%%%%%%%%%%%%%%%%%%%%%%%%%%%%%%%%%%%%%%%%%%%
\subsection{Where can I get news?}
\label{intro.news}

News and updates can be obtained from the the Cheetah website:
\url{http://CheetahTemplate.sourceforge.net}

Cheetah discussions take place on the list
\email{cheetahtemplate-discuss@lists.sourceforge.net}.

If you encounter difficulties, or are unsure about how to do something,
please post a detailed message to the list.

%%%%%%%%%%%%%%%%%%%%%%%%%%%%%%%%%%%%%%%%%%%%%%%%%%%%%%%%%%%%%%%%%%%%%%%%%%%%%%%%
\subsection{How can I contribute?}
\label{intro.contribute}

Cheetah is the work of many volunteers.  If you use Cheetah please share your
experiences, tricks, customizations, and frustrations.

\subsubsection{Bug reports and patches}

If you think there is a bug in Cheetah, send a message to the email list
with the following information:

\begin{enumerate}
\item a description of what you were trying to do and what happened
\item all tracebacks and error output
\item your version of Cheetah
\item your version of Python
\item your operating system
\item whether you have changed anything in the Cheetah installation
\end{enumerate}

\subsubsection{Example sites and tutorials}
If you're developing a website with Cheetah, please send a link to the
email list so we can keep track of Cheetah sites.  Also, if you discover
new and interesting ways to use Cheetah please share your experience and
write a quick tutorial about your technique.

\subsubsection{Macro libraries}
We hope to build up a framework of macros libraries (see section
\ref{macros.libraries}) to distribute with Cheetah and would appreciate
any contributions.

\subsubsection{Test cases}
Cheetah is packaged with a regression testing suite that is run with each
new release to ensure that everything is working as expected and that recent
changes haven't broken anything.  The test cases are in the Cheetah.Tests
module.  If you find a reproduceable bug please consider writing a test case
that will pass only when the bug is fixed.  Send any new test cases to the email
list with the subject-line ``new test case for Cheetah.''

\subsubsection{Publicity}
Help spread the word ... recommend it to others, write articles about it, etc.

%%%%%%%%%%%%%%%%%%%%%%%%%%%%%%%%%%%%%%%%%%%%%%%%%%%%%%%%%%%%%%%%%%%%%%%%%%%%%%%%
\subsection{Acknowledgements}
\label{intro.acknowledgments}
    
We'd like to thank the following people for contributing valuable advice, code
and encouragement: Geoff Talvola, Jay Love, Terrel Shumway, Sasa Zivkov, Arkaitz
Bitorika, Jeremiah Bellomy, Baruch Even, Paul Boddie, Stephan Diehl, and Geir
Magnusson.
    
The Velocity, WebMacro and Smarty projects provided inspiration and design
ideas.  Cheetah has benefited from the creativity and energy of their
developers. Thank you.

%%%%%%%%%%%%%%%%%%%%%%%%%%%%%%%%%%%%%%%%%%%%%%%%%%%%%%%%%%%%%%%%%%%%%%%%%%%%%%%%
\subsection{License}
\label{intro.license}

Cheetah is released for unlimited distribution under the terms of the
Python license.

