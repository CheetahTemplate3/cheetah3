\section{Putting it all together}
\label{puttingItTogether}

This section provides a high-level explanation of how to use Cheetah and an
under-the-hood view of how Cheetah works.  If you are using Cheetah with Webware
and want to get straight into it, skip ahead to section
\ref{webware.inheritance.tmpl} and come back here later.

%%%%%%%%%%%%%%%%%%%%%%%%%%%%%%%%%%%%%%%%%%%%%%%%%%%%%%%%%%%%%%%%%%%%%%%%%%%%%%%%
\subsection{The Template class}
\label{puttingItTogether.template}

The \code{Template} class (\code{Cheetah.Template.Template}) is the heart of
Cheetah.  Objects of this class parse and compile template definitions into a
chunk of Python code.  This chunk of Python code is then bound to the template
object's \code{.__str__()} and \code{.respond()} methods, and is executed each
time you request the filled output from the template by calling either of these
methods.  

%%%%%%%%%%%%%%%%%%%%%%%%%%%%%%%%%%%%%%%%%%%%%%%%%%%%%%%%%%%%%%%%%%%%%%%%%%%%%%%%
\subsubsection{Constructing template objects}
\label{puttingItTogether.template.constructing}

Template objects may be created manually in Python code or automatically using
\code{.tmpl} files when Cheetah is used with Webware. See section
\ref{webware.inheritance.tmpl} for more information on \code{.tmpl} files.  The
section only deals with manual creation.

You must pass either a Template Definition string or a 'file' keyword argument,
but not both, to the \code{Template} class' constructor.

Here are some examples of the various ways to create a template object:
\begin{description}
\item{\code{templateObj = Template("The king is a \$placeholder1.")}}
     Pass the Template Definition as a string.
\item{\code{templateObj = Template(file="fink.ctd")}}
     Read the Template Definition from a file named "fink.ctd".  Note, do not
     confuse \code{.ctd} files with the \code{.tmpl} files that may be used
     directly with Webware (see section \ref{webware.inheritance.tmpl})
\item{\code{templateObj = Template(file=f)}}
     Read the Template Definition from file-like object 'f'.
\item{\code{templateObj = Template("The king is a \$placeholder1.", dict, obj)}}
     Pass the Template Definition as a string.  Also pass two Namespaces for the
     searchList: a dictionary 'dict' and an instance 'obj'.
\item{\code{templateObj = Template(None, dict, obj, file="fink.txt")}}
     Same, but pass a filename instead of a string.  The \code{None} is required
     here to represent the missing Template Definition string -- this due to
     Python's rules for positional parameters.
\item{\code{templateObj = Template(None, dict, obj, file="fink.txt")}}
     Same with a file object.
\end{description}

The following usage examples are not allowed:
\begin{verbatim}
templateObj = Template() 
templateObj = Template("The king is a $placeholder1", file="fink.txt")
templateObj = Template("The king is a $placeholder1", file=f)
\end{verbatim}

%%%%%%%%%%%%%%%%%%%%%%%%%%%%%%%%%%%%%%%%%%%%%%%%%%%%%%%%%%%%%%%%%%%%%%%%%%%%%%%%
\subsubsection{Using Template objects}
\label{puttingItTogether.template.using}

To use a template object once it has been created, simply print it using
Python's \code{print} keyword, apply the \code{str()} function to it, or call
its \code{.respond} method.  The result is a string with all the Placeholders
filled in with their current values, and all the directives executed.

%%%%%%%%%%%%%%%%%%%%%%%%%%%%%%%%%%%%%%%%%%%%%%%%%%%%%%%%%%%%%%%%%%%%%%%%%%%%%%%%
\subsubsection{Methods of the Template class}
\label{puttingItTogether.template.methods}

\begin{description}
     
\item[.compileTemplate()] -- Compile the template.  This needs to be called
     manually if the setting ``delayedCompile'' is set to True.
     
\item[.recompile()] -- Synonym for \code{.compileTemplate}.
     
\item[.searchList()] -- Return a reference to the searchList.  It's a
     \code{UserList} subclass, so you can use Python's standard list operations
     on it.
     
\item[.addToSearchList(theNamespace)] -- Append the given Namespace to the
     searchList.
     
\item[.extendTemplate(templateExt)] --

\item[.redefineTemplateBlock(blockName, blockContents)] -- Redefine a named block
     in the template definition.  Both arguments must be strings.  You must call
     \code{.recompile()} for the change to take effect if the template has
     already been compiled.
     
\item[.killTemplateBlock(*blockNames)] Delete the named blocks from the template
     definition.  They will no longer show up in Filled Templates. You must call
     \code{.recompile()} for the change to take effect if the template has
     already been compiled.

     
\item[.loadMacro(), .loadMacros() and .loadMacrosFromModule()] See section
     \ref{directives.macros}.

\item[.registerServerPlugin(plugin)]  Register a plugin that extends the
     functionality of the Template.  See section \ref{customizing.plugins}.

\item[.normalizePath()] A hook to enable proper handling of server-side paths
     with Webware.

\item[.getFileContents(fileName)]  Returns the contents of the named
     file.  May be overridded to do, e.g., URL retrievals.

\item[.runAsMainProgram()]

\end{description}

Since the Template class inherits from the \code{SettingsManager} mixin class,
its methods are available to change configuration settings.  See section
\ref{libraries.SettingsManager} for more about it.


%%%%%%%%%%%%%%%%%%%%%%%%%%%%%%%%%%%%%%%%%%%%%%%%%%%%%%%%%%%%%%%%%%%%%%%%%%%%%%%%
\subsection{Putting stuff into the searchList}
\label{puttingItTogether.searchList}

%%%%%%%%%%%%%%%%%%%%%%%%%%%%%%%%%%%%%%%%%%%%%%%%%%%%%%%%%%%%%%%%%%%%%%%%%%%%%%%%
\subsubsection{As attributes/methods of the Template object}
\label{puttingItTogether.searchList.self}

%%%%%%%%%%%%%%%%%%%%%%%%%%%%%%%%%%%%%%%%%%%%%%%%%%%%%%%%%%%%%%%%%%%%%%%%%%%%%%%%
\subsubsection{Using the \#data directive}
\label{puttingItTogether.searchList.dataDirective}

%%%%%%%%%%%%%%%%%%%%%%%%%%%%%%%%%%%%%%%%%%%%%%%%%%%%%%%%%%%%%%%%%%%%%%%%%%%%%%%%
\subsubsection{Using the .addToSearchList() method}
\label{puttingItTogether.searchList.addToSearchList}


%%%%%%%%%%%%%%%%%%%%%%%%%%%%%%%%%%%%%%%%%%%%%%%%%%%%%%%%%%%%%%%%%%%%%%%%%%%%%%%%
\subsection{Optimization}
\label{puttingItTogether.optimization}

%%%%%%%%%%%%%%%%%%%%%%%%%%%%%%%%%%%%%%%%%%%%%%%%%%%%%%%%%%%%%%%%%%%%%%%%%%%%%%%%
\subsection{Debugging}
\label{puttingItTogether.debugging}



