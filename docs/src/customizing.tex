\section{Customizing and extending Cheetah}
\label{customizing}

Template objects have a \code{\_settings} attribute. This is a dictionary of
configuration settings that control the core behaviour of Cheetah.  The Template
object's constructor method has a keyword argument 'settings' that accepts a
dictionary to override the default settings.  Many of these settings are for
internal use and are of little interest to end-users.  Some, however, can be
quite useful. This chapter explains how to use them.

%%%%%%%%%%%%%%%%%%%%%%%%%%%%%%%%%%%%%%%%%%%%%%%%%%%%%%%%%%%%%%%%%%%%%%%%%%%%%%%%
\subsection{Unknown placeholder names}
%% handlers

Here's 
Example:

\begin{verbatim}
from Cheetah.Template import Template
templateObj = Template(myTemplateDef, myNamespace, 
                 settings={'useAutocalling':0,
                           'placeholderStartToken':'$'})
\end{verbatim}


%%%%%%%%%%%%%%%%%%%%%%%%%%%%%%%%%%%%%%%%%%%%%%%%%%%%%%%%%%%%%%%%%%%%%%%%%%%%%%%%
\subsection{Unknown placeholder names}
%% handlers

The 'varNotFound\_handler' setting controls what happens when a Placeholder
Name is not found in the Search List.  The default setting simply places the
Placeholder Tag in the output.  The ``Big Warning'' handler puts some
harder-to-miss decorations around it.  To switch to the Big Warning handler:

\begin{verbatim}
from Cheetah import CodeGenerator, Template
templateObj = Template(myTemplateDef, 
        varNotFound_handler=CodeGenerator.varNotFound\_echo)
# Of course, you can add any number of Namespaces after myTemplateDef.
\end{verbatim}

Defining your own handler is easy too, as you can see from the reprint of
Cheetah's handlers below.  Just define a function that takes two arguments, 
(1) the current TO and (2) the Placeholder Name as a string.  Return whatever
you want Cheetah to insert as the Placeholder Value.

\begin{verbatim}
## varNotFound handlers ##
def varNotFound\_echo(templateObj, tag):
    return "$" + tag

def varNotFound_bigWarning(templateObj, tag):
    return "="*15 + "&lt;$" + tag + " could not be found&gt;" + "="*15
    
\end{verbatim}

% Not committed yet:
%def varNotFound_KeyError(templateObj, tag):
%    raise KeyError("no '%s' in this Template Object's Search List" % tag)

% You can also get a list of Placeholder Names missing in the Search List:
% \code{templateObj.getUnknowns()}.

%% default vars

%%%%%%%%%%%%%%%%%%%%%%%%%%%%%%%%%%%%%%%%%%%%%%%%%%%%%%%%%%%%%%%%%%%%%%%%%%%%%%%%
\subsection{Plugins}
\subsubsection{The PSP plugin}
Cheetah ships with a plugin that enables pure PSP-style code to be used along
with Cheetah \$placeholders and \#directives.  PSP syntax can also be used
alone without \$placeholders and \#directives.

Here's a trivial example of how to use it:
\begin{verbatim}
from Cheetah.Template import Cheetah
from Cheetah.Plugins.PSP import PSPplugin

templateDef = """
Testing Cheetah's PSP plugin:
 
$testVar
<% pspVar = 'X' %>
#set $list = [1,2,3]
 
#for $i in map(lambda x: x*x*x, $list)
$i
<%for j in range(15):%> <%=j*15%><%=pspVar%><%end%>
#end for
"""
print Template(templateDef, {'testVar':1234}, plugins=[PSPplugin()])

\end{verbatim}

%%%%%%%%%%%%%%%%%%%%%%%%%%%%%%%%%%%%%%%%%%%%%%%%%%%%%%%%%%%%%%%%%%%%%%%%%%%%%%%%
\subsection{Custom variable-tags}

%%%%%%%%%%%%%%%%%%%%%%%%%%%%%%%%%%%%%%%%%%%%%%%%%%%%%%%%%%%%%%%%%%%%%%%%%%%%%%%%
\subsection{Custom directives}
