\section{Generating, Caching and Filtering Output}
\label{output}

%%%%%%%%%%%%%%%%%%%%%%%%%%%%%%%%%%%%%%%%%%%%%%%%%%%%%%%%%%%%%%%%%%%%%%%%%%%%%%%%
\subsection{Output from simple expressions: placeholders}
\label{output.placeholders}

Placeholders are like 'primaries' in Python (see section 5.3 of Python's
Language Reference for more). If you don't know what a 'primary' is, think of
them as variables. They may refer to any Python value: strings, numbers, None,
tuples, tuple items, tuple slices, lists, list items, list slices, dictionaries,
dictionary values, objects, and the return values of function and method calls.
When a template is processed each placeholder will be replaced with the string
representation of its value.  These string representations are identical to
those returned by Python's str() function, with the exception of \code{None}.
\code{None} is replaced with an empty string, or in other words nothing.
(This is with the default output filter.  You can customize the string
representation via the \code{\#filter} directive, section \ref{output.filter}.)

Cheetah finds the values for placeholders in several places.  The most obvious
place is in the searchList.  But placeholders can also refer to local variables
(created by \code{\#set}), global variables (created by \code{\#set global}, 
\code{\#import}, methods of the generated class (created by 
\code{\#def} and \code{\#block} or inherited from the \code{Template} class),
attributes of the generated class (created by \code{\#attr} or inherited), or
Python builtins like \code{\$range} or \code{\$len}.  

%%%%%%%%%%%%%%%%%%%%%%%%%%%%%%%%%%%%%%%%%%%%%%%%%%%%%%%%%%%%%%%%%%%%%%%%%%%%%%%%
\subsection{Output from complex expressions: \#echo}
\label{output.echo}

The \code{\#echo} directive is used to echo the output from expressions that
can't be written as simple \$placeholders.  It serves the same purpose as ASP's
$<$\%=EXPR\%$>$ tag.

\begin{verbatim}
Here is my #echo ', '.join(['silly']*5) # example 
\end{verbatim}

This produces:

\begin{verbatim}
Here is my silly, silly, silly, silly, silly example.
\end{verbatim}

In this example, the second \code{\#} serves to end the \code{\#echo} directive,
allowing plain text to follow it.

%%%%%%%%%%%%%%%%%%%%%%%%%%%%%%%%%%%%%%%%%%%%%%%%%%%%%%%%%%%%%%%%%%%%%%%%%%%%%%%%
\subsection{Executing expressions without output: \#silent}
\label{output.silent}

\code{\#silent} is the opposite of \code{\#echo}.  It executes an expression
but discards the output.  It's similar to ASP's $<$\% EXPR \%$>$ tag.

\begin{verbatim}
#silent $myList.reverse()
#silent $myList.sort()
Here is #silent $covertOperation() # nothing
\end{verbatim}

If your template requires some Python code to be executed at the beginning;
(e.g., to calculate placeholder values, access a database, etc), you can put
it in a "doEverything" method you inherit, and call this method using
\code{\#silent} at the top of the template.

%%%%%%%%%%%%%%%%%%%%%%%%%%%%%%%%%%%%%%%%%%%%%%%%%%%%%%%%%%%%%%%%%%%%%%%%%%%%%%%%
\subsection{Caching Output}
\label{output.caching}

%%%%%%%%%%%%%%%%%%%%%%%%%%%%%%%%%%%%%%%%%%%%%%%%%%%%%%%%%%%%%%%%%%%%%%%%%%%%%%%%
\subsubsection{Caching individual placeholders}
\label{output.caching.placeholders}

By default, the values of each \$placeholder is retrieved and
interpolated for every request. However, it's possible to cache the values
of individual placeholders if they don't change very often, in order to 
speed up the template filling.
                         
To cache the value of a single \code{\$placeholder}, add an asterisk after the
\$; e.g.,  \code{\$*var}.  The first time the template is
filled, \code{\$var} is looked up.  Then whenever the template is filled again,
the cached value is used instead of doing another lookup.  

The \code{\$*} format caches ``forever''; that is, as long as the template
instance remains in memory.  It's also possible to cache for a certain time
period using the form \code{\$*<interval>*variable}, where \code{<interval>} is
the interval.  The time interval can be specified in seconds (5s), minutes
(15m), hours (3h), days (2d) or weeks (1.5w). The default is minutes.

\begin{verbatim}
<HTML>
<HEAD><TITLE>$title</TITLE></HEAD>
<BODY>

$var ${var}           ## dynamic - will be reinterpolated for each request
$*var2 $*{var2}       ## static - will be interpolated only once at start-up
$*5*var3 $*5*{var3}   ## timed refresh - will be updated every five minutes.

</BODY>
</HTML>
\end{verbatim}

Note that ``every five minutes'' in the example really means every five
minutes: the variable is looked up again when the time limit is reached,
whether the template is being filled that frequently or not.  Keep this in
mind when setting refresh times for CPU-intensive or I/O intensive 
operations.

If you're using the long placeholder syntax, \verb+${}+, the braces go only
around the placeholder name: \verb+$*.5h*{var.func('arg')}+.

Sometimes it's preferable to explicitly invalidate a cached item whenever
you say so rather than at certain time intervals.  You can't do this with
individual placeholders, but you can do it with cached regions, which will
be described next.

%%%%%%%%%%%%%%%%%%%%%%%%%%%%%%%%%%%%%%%%%%%%%%%%%%%%%%%%%%%%%%%%%%%%%%%%%%%%%%%%
\subsubsection{Caching entire regions}
\label{output.caching.regions}

The \code{\#cache \ldots \#end cache} directive is used to cache a region of
content in a template.  It accepts several keyword arguments as demonstrated in
the examples below.  Cached regions can be given an 'id', which can be used to
programmatically refresh a region with the \code{.refreshCache(id)}
method.  Cached regions can also be invalidated according to a user-specified
\code{test} conditions.

\begin{verbatim}
#cache timer='30m', ID='cache1'
#for $cust in $customers
$cust.name:
$cust.street - $cust.city
#end for
#end cache
\end{verbatim}

\begin{verbatim}
#cache id='sidebar', test=$isDBUpdated
## do something
#end cache

#cache id='sidebar2', test=($isDBUpdated or $someOtherCondition)
## do something
#end cache
\end{verbatim}

Note that placeholders lose their identity inside a \code{\#cache} region.
That is, we do not cache \code{\$cust.name} and the \code{\#for} loop above
individually.  Instead, the region is placed in the cache as plain text
{\em after} the placeholders and directives have been evaluated.  When/if
the cached item is invalidated, the placeholders and directives are evaluated
again.  A corollary of this is that placeholders inside \code{\#cache}
regions cannot use the \code{\$*var} or \code{\$*<interval>*var} form,
because they cannot be cached under a different policy than the region 
they're in.

The \code{\#cache} directive cannot be nested.

We are planning to add a \code{'varyBy'} keyword argument in the future that
will allow a separate cache instances to be created for a variety of conditions,
such as different query string parameters or browser types. This is inspired by
ASP.net's varyByParam and varyByBrowser output caching keywords.

% @@MO: Can we cache by Webware sessions?  What about sessions where the
% session ID is encoded as a path prefix in the URI?  Need examples.

% @@MO: Are cache parameters case sensitive?  If so, correct "id" and "ID"
% above.

%%%%%%%%%%%%%%%%%%%%%%%%%%%%%%%%%%%%%%%%%%%%%%%%%%%%%%%%%%%%%%%%%%%%%%%%%%%%%%%%
\subsection{\#raw}
\label{output.raw}

Any section of a template definition that is inside a \code{\#raw \ldots
\#end raw} tag pair will be printed verbatim without any parsing of
\$placeholders or other directives. This can be very useful for debugging, or
for Cheetah examples and tutorials.

\code{\#raw} is conceptually similar to HTML's \code{<PRE>} tag and LaTeX's
\code{\\verbatim\{\}} tag, but unlike those tags, \code{\#raw} does not cause
the body to appear in a special font or typeface.  It can't, because Cheetah
doesn't what a font is.  


%%%%%%%%%%%%%%%%%%%%%%%%%%%%%%%%%%%%%%%%%%%%%%%%%%%%%%%%%%%%%%%%%%%%%%%%%%%%%%%%
\subsection{\#include}
\label{output.include}

The \code{\#include} directive is  used to include text from outside the
template definition.  The text can come from an external file or from a
\code{\$placeholder} variable.  When working with external files, Cheetah will
monitor for changes to the included file and update as necessary.  

This example demonstrates its use with external files:
\begin{verbatim}
#include "includeFileName.txt"
\end{verbatim}
The content of "includeFileName.txt" will be parsed for Cheetah syntax.

And this example demonstrates use with \code{\$placeholder} variables:
\begin{verbatim}
#include source=$myParseText
\end{verbatim}
The value of \code{\$myParseText} will be parsed for Cheetah syntax. This is not
the same as simple placing the \$placeholder tag ``\code{\$myParseText}'' in
the template definition.  In the latter case, the value of \$myParseText would
not be parsed.

By default, included text will be parsed for Cheetah tags.  The argument
``\code{raw}'' can be used to suppress the parsing.

\begin{verbatim}
#include raw "includeFileName.txt"
#include raw source=$myParseText
\end{verbatim}

Cheetah wraps each chunk of \code{\#include} text inside a nested
\code{Template} object.  Each nested template has a copy of the main
template's searchList.  However, \code{\#set} variables are visible
across includes only if the defined using the \code{\#set global} keyword.

All directives must be balanced in the include file.  That is, if you start
a \code{\#for} or \code{\#if} block inside the include, you must end it in
the same include.  (This is unlike PHP, which allows unbalanced constructs
in include files.)

% @@MO: What did we decide about #include and the searchList?  Does it really
% use a copy of the searchList, or does it share the searchList with the
% parent?

% @@MO: deleted
%These nested templates share the same \code{searchList}
%as the top-level template. 

%%%%%%%%%%%%%%%%%%%%%%%%%%%%%%%%%%%%%%%%%%%%%%%%%%%%%%%%%%%%%%%%%%%%%%%%%%%%%%%%
\subsection{\#slurp}
\label{output.slurp}

The \code{\#slurp} directive eats up the trailing newline on the line it
appears in, joining the following line onto the current line.


It is particularly useful in \code{\#for} loops:
\begin{verbatim}
#for $i in range(5)
$i #slurp
#end for
\end{verbatim}
outputs:
\begin{verbatim}
0 1 2 3 4
\end{verbatim}


%%%%%%%%%%%%%%%%%%%%%%%%%%%%%%%%%%%%%%%%%%%%%%%%%%%%%%%%%%%%%%%%%%%%%%%%%%%%%%%%
\subsection{Ouput Filtering and \#filter}
\label{output.filter}

Output from \$placeholders is passed through an ouput filter.  The default
filter merely returns a string representation of the placeholder value,
unless the value is \code{None}, in which case the filter returns an empty
string. 

Certain filters take optional arguments to modify their behaviour.  To pass
arguments, use the long placeholder syntax and precede each filter argument by
a comma.  Filter arguments don't take a \code{\$} prefix, to avoid clutter in
the placeholder tag which already has plenty of dollar signs.  For instance,
the MaxLen filter takes an argument 'maxlen':

\begin{verbatim}
${placeholderName, maxlen=20}
${functionCall($functionArg), maxlen=$myMaxLen}
\end{verbatim}

To change the output filter, use the \code{'filter'} keyword to the
\code{Template} class constructor, or the \code{\#filter}
directive at runtime (details below).  You may use \code{\#filter} as often as
you wish to switch between several filters, if certain \code{\$placeholders}
need one filter and other \code{\$placeholders} need another.

The standard filters are in the module \code{Cheetah.Filters}.  Cheetah
currently provides:

\begin{description}
\item{\code{Filter}}
     \\ The default filter, which converts None to '' and everything else to
     \code{str(whateverItIs)}.  This is the base class for all other filters,
     and the minimum behaviour for all filters distributed with Cheetah.
\item{\code{ReplaceNone}}
     \\ Same.
\item{\code{MaxLen}}
     \\ Same, but truncate the value if it's longer than a certain length.
     Use the 'maxlen' filter argument to specify the length, as in the
     examples above.  If you don't specify 'maxlen', the value will not be
     truncated.
\item{\code{Pager}}
     \\ Output a "pageful" of a long string.  After the page, output HTML
     hyperlinks to the previous and next pages.  This filter uses several
     filter arguments and environmental variables, which have not been 
     documented yet.
\item{\code{WebSafe}}
     \\ Same as default, but convert HTML-sensitive characters ('<', '\&', '>')
     to HTML entities so that the browser will display them literally rather
     than interpreting them as HTML tags.  This is useful with database values
     or user input that may contain sensitive characters.  But if your values
     contain embedded HTML tags you want to preserve, you do not want this 
     filter.
     
     The filter argument 'also' may be used to specify additional characters to
     escape.  For instance, say you want to ensure a value displays all on one
     line.  Escape all spaces in the value with '\&nbsp', the non-breaking
     space:
\begin{verbatim}
${$country, $also=' '}}
\end{verbatim}
\end{description}

To switch filters using a class object, pass the class using the
{\bf filter} argument to the Template constructor, or via a placeholder to the
\code{\#filter} directive: \code{\#filter \$myFilterClass}.  The class must be
a subclass of \code{Cheetah.Filters.Filter}.  When passing a class object, the
value of {\bf filtersLib} does not matter, and it does not matter where the
class was defined.

To switch filters by name, pass the name of the class as a string using the
{\bf filter} argument to the Template constructor, or as a bare word (without
quotes) to the \code{\#filter} directive: \code{\#filter TheFilter}.  The
class will be looked up in the {\bf filtersLib}.

The filtersLib is a module containing filter classes, by default
\code{Cheetah.Filters}.  All classes in the module that are subclasses of
\code{Cheetah.Filters.Filter} are considered filters.  If your filters are in
another module, pass the module object as the {\bf filtersLib} argument to the
Template constructor.

You can always switch back to the default filter this way:
\code{\#filter None}.  This is easy to remember because "no filter" means the
default filter, and because None happens to be the only object the default
filter treats specially.

We are considering additional filters; see
\url{http://webware.colorstudy.net/twiki/bin/view/Cheetah/MoreFilters}
for the latest ideas.

%% @@MO: Is '#end filter' implemented?  Will it be?  Can filters nest?
%% Will '#end filter' and '#filter None' be equivalent?

%% @@MO: Tavis TODO: fix the description of the Pager filter.  It needs a howto.

%% @@MO: How about using settings to provide default arguments for filters?
%% Each filter could look up FilterName (or FilterNameDefaults) setting,
%% whose value would be a dictionary containing keyword/value pairs.  These
%% would be overridden by same-name keys passed by the placeholder.

%% @@MO: If sed-filters (#sed) get added to Cheetah, give them a section here.

% Local Variables:
% TeX-master: "users_guide"
% End:      

